\documentclass{article}
\usepackage{graphicx}
\usepackage{amsmath}
\usepackage{amsfonts}
\usepackage{mathtools}
\usepackage[margin=1.5cm]{geometry}
\begin{document}

\section {Appendix}

Some of the derivations in this document feature element-wise functions that need to be applied over each individual element in a vector/matrix, particularly as a function of that element's row and/or column. A few fields use some of their own notation for this kind of math, but overall, there seems to be no standardized notation for such functions. This appendix has customized notation which I personally came up with in order to not have to waste a lot of space trying to explain the equations in here. \newline

Take a generic function $f(x)$. The column vector $\vec{v}$ (populated by applying $f(x)$ element-wise to each row in the vector as a function of the row) can be denoted below as such:

\begin{align*}
    \vec{v} = \begin{bmatrix} f(n) \end{bmatrix}^{N}
            = \begin{bmatrix} f(n) \end{bmatrix}^{n=N - 1}_{n=0}
            = \begin{bmatrix}
                    f(N - 1) \\ \\
                    f(N - 2) \\ \\
                    f(N - 3) \\ \\
                      \vdots \\ \\
                    f(    2) \\ \\
                    f(    1) \\ \\
                    f(    0) 
                \end{bmatrix}
  \end{align*}

Accordingly, the row vector $\vec{v}^{T}$ can be denoted below as such:

\begin{align*}
    \vec{v}^{T}&=   \prescript{}{N}{\begin{bmatrix} f(n) \end{bmatrix}} \ 
                = \ \prescript{}{n=N - 1}{\begin{bmatrix} f(n) \end{bmatrix}}_{n=0} \\ \\
               &= \begin{bmatrix} f(N - 1) & f(N - 2) & f(N - 3) & \ldots & f(2) & f(1) & f(0) \end{bmatrix} \\
  \end{align*} 

The matrix $M$ (populated by applying $f(x)$ elementwise to each row and column in the matrix as a function of the row and column) can be denoted below as such:

\begin{align*}
    M&=   \prescript{}{N}{\begin{bmatrix} f(n, m) \end{bmatrix}}^{N} \ 
      = \ \prescript{}{m = N - 1}{\begin{bmatrix} f(n, m) \end{bmatrix}}^{n = N - 1}_{n = 0,m =0} \\ \\
     &= \begin{bmatrix}
            f(N - 1, N - 1) & f(N - 1, N - 2) & f(N - 1, N - 3) & \ldots & f(N - 1, 2) & f(N - 1, 1) & f(N - 1, 0) \\ \\
            f(N - 2, N - 1) & f(N - 2, N - 2) & f(N - 2, N - 3) & \ldots & f(N - 2, 2) & f(N - 2, 1) & f(N - 2, 0) \\ \\
            f(N - 3, N - 1) & f(N - 3, N - 2) & f(N - 3, N - 3) & \ldots & f(N - 3, 2) & f(N - 3, 1) & f(N - 3, 0) \\ \\
                     \vdots &          \vdots &          \vdots & \ddots &      \vdots &      \vdots &      \vdots \\ \\
            f(    2, N - 1) & f(    2, N - 2) & f(    2, N - 3) & \ldots & f(    2, 2) & f(    2, 1) & f(    2, 0) \\ \\
            f(    1, N - 1) & f(    1, N - 2) & f(    1, N - 3) & \ldots & f(    1, 2) & f(    1, 1) & f(    1, 0) \\ \\
            f(    0, N - 1) & f(    0, N - 2) & f(    0, N - 3) & \ldots & f(    0, 2) & f(    0, 1) & f(    0, 0)
          \end{bmatrix} \\ \\
  \end{align*} 

\newpage

Sometimes, some matrices will have portions systematically populated by zeroes. In such cases, we can have the original function ($f(x)$) nested within another function ($g(x)$) which acts as a sort of indicator function in order to keep this notation short and brief. For example, consider an upper triangular matrix $U$. $U$ can be denoted as such below:

\begin{align*}
    U&= \prescript{}{N}{\begin{bmatrix} g(n, m) \end{bmatrix}}^{N} \quad , \quad g(n, m) = \begin{cases}
                                                                                                            f(n, m), \quad n \geq m \\ \\
                                                                                                \quad \quad \quad 0, \quad n < m
                                                                                             \end{cases} \\ \\
     &= \begin{bmatrix}
            f(N - 1, N - 1) & f(N - 1, N - 2) & f(N - 1, N - 3) & \ldots & f(N - 1, 2) & f(N - 1, 1) & f(N - 1, 0) \\ \\
                          0 & f(N - 2, N - 2) & f(N - 2, N - 3) & \ldots & f(N - 2, 2) & f(N - 2, 1) & f(N - 2, 0) \\ \\
                          0 &               0 & f(N - 3, N - 3) & \ldots & f(N - 3, 2) & f(N - 3, 1) & f(N - 3, 0) \\ \\
                     \vdots &          \vdots &          \vdots & \ddots &      \vdots &      \vdots &      \vdots \\ \\
                          0 &               0 &               0 & \ldots & f(    2, 2) & f(    2, 1) & f(    2, 0) \\ \\
                          0 &               0 &               0 & \ldots &           0 & f(    1, 1) & f(    1, 0) \\ \\
                          0 &               0 &               0 & \ldots &           0 &           0 & f(    0, 0)
          \end{bmatrix} \\ \\
  \end{align*}

The inner product of $\vec{v}$ with itself can be denoted as such below:

\begin{align*}
    \vec{v}^{T}\vec{v} = \prescript{}{N}{\begin{bmatrix} f(n) \end{bmatrix}}\begin{bmatrix} f(n) \end{bmatrix}^{N}
                       = \prescript{}{n = N - 1}{\begin{bmatrix} f(n) \end{bmatrix}}_{n = 0}\begin{bmatrix} f(n) \end{bmatrix}^{n = N - 1}_{n = 0} = \sum^{n = N - 1}_{n = 0}f^{2}(n) \\ \\
  \end{align*}

The outer product of $\vec{v}$ with itself can be denoted as such below:

\begin{align*}
    \vec{v}\vec{v}^{T}&= \begin{bmatrix} f(n) \end{bmatrix}^{N}\prescript{}{N}{\begin{bmatrix} f(n) \end{bmatrix}}
                       = \begin{bmatrix} f(n) \end{bmatrix}^{n = N - 1}_{n = 0}\prescript{}{n = N - 1}{\begin{bmatrix} f(n) \end{bmatrix}}_{n = 0}
                       = \prescript{}{m = N - 1}{\begin{bmatrix} f(n)f(m) \end{bmatrix}}^{n = N - 1}_{n = 0, m = 0} \\ \\ 
                      &= \begin{bmatrix}
                            f(N - 1)f(N - 1) & f(N - 1)f(N - 2) & f(N - 1)f(N - 3) & \ldots & f(N - 1)f(2) & f(N - 1)f(1) & f(N - 1)f(0) \\ \\
                            f(N - 2)f(N - 1) & f(N - 2)f(N - 2) & f(N - 2)f(N - 3) & \ldots & f(N - 2)f(2) & f(N - 2)f(1) & f(N - 2)f(0) \\ \\
                            f(N - 3)f(N - 1) & f(N - 3)f(N - 2) & f(N - 3)f(N - 3) & \ldots & f(N - 3)f(2) & f(N - 3)f(1) & f(N - 3)f(0) \\ \\
                                      \vdots &           \vdots &           \vdots & \ddots &       \vdots &       \vdots &       \vdots \\ \\
                            f(    2)f(N - 1) & f(    2)f(N - 2) & f(    2)f(N - 3) & \ldots & f(    2)f(2) & f(    2)f(1) & f(    2)f(0) \\ \\
                            f(    1)f(N - 1) & f(    1)f(N - 2) & f(    1)f(N - 3) & \ldots & f(    1)f(2) & f(    1)f(1) & f(    1)f(0) \\ \\
                            f(    0)f(N - 1) & f(    0)f(N - 2) & f(    0)f(N - 3) & \ldots & f(    0)f(2) & f(    0)f(1) & f(    0)f(0)
                           \end{bmatrix} \\ \\ 
  \end{align*}

\end{document}
