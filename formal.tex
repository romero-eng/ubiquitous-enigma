\documentclass{article}
\usepackage{graphicx}
\usepackage{amsmath}
\usepackage{amsfonts}
\usepackage{mathtools}
\usepackage[margin=1.5cm]{geometry}
\begin{document}

\section{Main}

For an irreducible quadratic polynomial over the $\mathbb{Z}$-plane:

\begin{align*}
    & \quad \quad \quad f(z) = A\big(1 + \rho z^{-1}\big)\big(1 + \rho^{*}z^{-1}\big) \ , \quad \rho \in \mathbb{C} \\ \\
    & \Rightarrow \Big|f\big(\omega\big)\Big|^{2} = G^{2}\Big[\cos\big(\omega\big) + \Gamma\Big]\Big[\cos\big(\omega\big) + \Gamma^{*}\Big] \ , \quad \Gamma \in \mathbb{C} \\ \\ 
    & \quad \quad \quad \quad \quad \quad \quad \quad \text{where} \ G \equiv 2A\big|\rho\big| \\ \\
    & \quad \quad \quad \quad \quad \quad \quad \quad \quad \quad \quad \Gamma_{R} \equiv \cos\big(\angle{\rho}\big)\cosh\Big(\ln\big[\big|\rho\big|\big]\Big) \\ \\  
    & \quad \quad \quad \quad \quad \quad \quad \quad \quad \quad \quad \Gamma_{I} \equiv \sin\big(\angle{\rho}\big)\sinh\Big(\ln\big[\big|\rho\big|\big]\Big) \\ \\
    & \Rightarrow \eta = \big|\Gamma\big|^{2} + \sqrt{\big|\Gamma\big|^{4} - 2\cos\big(2\angle{\Gamma}\big)|\Gamma|^{2} + 1} \\ \\
    & \Rightarrow \big|\rho\Big| = \sqrt{\eta - \sqrt{\eta^{2} - 1}} \\ \\
    & \Rightarrow \angle{\rho} = \bigg(\frac{1}{2}\bigg)\arccos\Big(2\big|\Gamma\big|^{2} - \eta\Big) \\ \\
    & \Rightarrow A = \frac{G}{2\big|\rho\big|}
  \end{align*}

\newpage
\section*{Appendix A: Irreversible Quadratic Spectral Magnitude Design}

For an irreducible quadratic polynomial over the $\mathbb{Z}$-plane:

\begin{align*}
    &\quad \quad \quad f(z) = \big(1 + \rho z^{-1}\big)\big(1 + \rho^{*}z^{-1}\big) \ , \quad \rho \in \mathbb{C} \\ \\
    &\Rightarrow \Big|f\big(\omega\big)\Big|^{2} = K^{2}\Big[\cos\big(\omega\big) + \Gamma\Big]\Big[\cos\big(\omega\big) + \Gamma^{*}\Big] \ , \quad \Gamma \in \mathbb{C} \\ \\ 
    & \quad \quad \quad \quad \quad \quad \quad \quad \text{where} \ K \equiv 2\big|\rho\big| \\ \\
    & \quad \quad \quad \quad \quad \quad \quad \quad \quad \quad \quad \Gamma_{R} \equiv \cos\big(\angle{\rho}\big)\cosh\Big(\ln\big[\big|\rho\big|\big]\Big) \\ \\  
    & \quad \quad \quad \quad \quad \quad \quad \quad \quad \quad \quad \Gamma_{I} \equiv \sin\big(\angle{\rho}\big)\sinh\Big(\ln\big[\big|\rho\big|\big]\Big)
  \end{align*}\newline

Effectively, the squared spectral magnitude can be expressed as an irreducible polynomial (over $\cos\big(\omega\big)$ instead of $x$) with $\Gamma$ and its complex conjugate acting as the roots for this polynomial.

\begin{align*}
    &\quad \ \ \big|\Gamma\big|^{2} = \Gamma_{R}^{2} + \Gamma_{I}^{2} \\ \\
    &\quad \quad \quad \ \ = \cos^{2}\big(\angle{\rho}\big)\cosh^{2}\Big(\ln\big[\big|\rho\big|\big]\Big) + \sin^{2}\big(\angle{\rho}\big)\sinh^{2}\Big(\ln\big[\big|\rho\big|\big]\Big) \\ \\
    &\quad \quad \quad \ \ = \cosh^{2}\Big(\ln\big[\big|\rho\big|\big]\Big) - \sin^{2}\big(\angle{\rho}\big) \\ \\
    &\quad \quad \quad \ \ = \bigg(\frac{1}{2}\bigg)\Bigg(\cosh\Big(\ln\big[\big|\rho\big|^{2}\big]\Big) + \cos\big(2\angle{\rho}\big)\Bigg) \\ \\
    &\Rightarrow 2\big|\Gamma\big| = \cosh\Big(\ln\big[\big|\rho\big|^{2}\big]\Big) + \cos\big(2\angle{\rho}\big)
  \end{align*}\newline

The equation above is one of two equations needed to express $\Gamma$ in terms of $\rho$. The second equation is derived down below:

\begin{align*}
    & \quad \ \ \cosh^{2}\Big(\ln\big[\big|\rho\big|\big]\Big) - \sinh^{2}\Big(\ln\big[\big|\rho\big|\big]\Big) = 1 \\ \\
    & \Rightarrow \frac{\Gamma_{R}^{2}}{\cos^{2}\big(\angle{\rho}\big)} - \frac{\Gamma_{I}^{2}}{\sin^{2}\big(\angle{\rho}\big)} = 1 \\ \\
    & \Rightarrow \sin^{2}\big(\angle{\rho}\big)\Gamma_{R}^{2} - \cos^{2}\big(\angle{\rho}\big)\Gamma_{I}^{2} = \cos^{2}\big(\angle{\rho}\big)\sin^{2}\big(\angle{\rho}\big) \\ \\
    & \Rightarrow \Gamma_{R}^{2} - \big|\Gamma\big|^{2}\cos^{2}\big(\angle{\rho}\big) = \cos^{2}\big(\angle{\rho}\big) - \cos^{4}\big(\angle{\rho}\big)
  \end{align*}

\newpage

\begin{align*}
    & \Rightarrow \Gamma_{R}^{2} - \big|\Gamma\big|^{2}\cos^{2}\big(\angle{\rho}\big) = \cos^{2}\big(\angle{\rho}\big) - \cos^{4}\big(\angle{\rho}\big) \\ \\
    & \Rightarrow \Bigg[\cos^{2}\big(\angle{\rho}\big) - \Bigg(\frac{\big|\Gamma\big|^{2} + 1}{2}\Bigg)\Bigg]^{2} = \Bigg(\frac{\big|\Gamma\big|^{2} + 1}{2}\Bigg)^{2} - \Gamma_{R}^{2} \\ \\ 
    & \Rightarrow \bigg[\cos\big(2\angle{\rho}\big) - \big|\Gamma\big|^{2}\bigg]^{2} = \big|\Gamma\big|^{4} - 2\cos\big(2\angle{\Gamma}\big)|\Gamma|^{2} + 1 \\ \\
    & \Rightarrow \bigg[\cos\big(2\angle{\rho}\big) - \big|\Gamma\big|^{2}\bigg]^{2} = \bigg[\big|\Gamma\big|^{2} - e^{2j\angle{\Gamma}}\bigg]\bigg[\big|\Gamma\big|^{2} - e^{-2j\angle{\Gamma}}\bigg] \\ \\
    & \Rightarrow \cos\big(2\angle{\rho}\big) = \big|\Gamma\big|^{2} \pm \sqrt{\bigg[\big|\Gamma\big|^{2} - e^{2j\angle{\Gamma}}\bigg]\bigg[\big|\Gamma\big|^{2} - e^{-2j\angle{\Gamma}}\bigg]} \\ \\
    & \Rightarrow \cos\big(2\angle{\rho}\big) = \big|\Gamma\big|^{2} - \sqrt{\bigg[\big|\Gamma\big|^{2} - e^{2j\angle{\Gamma}}\bigg]\bigg[\big|\Gamma\big|^{2} - e^{-2j\angle{\Gamma}}\bigg]}
  \end{align*}\newline

This second equation can then be plugged into the first equation to yield $\big|\rho\big|$ as a function of $\big|\Gamma\big|$ and $\angle{\Gamma}$:

\begin{align*}
    & \Rightarrow 2\big|\Gamma\big|^{2} = \cosh\Big(\ln\big[\big|\rho\big|^{2}\big]\Big) + \big|\Gamma\big|^{2} - \sqrt{\bigg[\big|\Gamma\big|^{2} - e^{2j\angle{\Gamma}}\bigg]\bigg[\big|\Gamma\big|^{2} - e^{-2j\angle{\Gamma}}\bigg]} \\ \\
    & \Rightarrow \cosh\Big(\ln\big[\big|\rho\big|^{2}\big]\Big) = \eta = \big|\Gamma\big|^{2} + \sqrt{\bigg[\big|\Gamma\big|^{2} - e^{2j\angle{\Gamma}}\bigg]\bigg[\big|\Gamma\big|^{2} - e^{-2j\angle{\Gamma}}\bigg]} \\ \\
    & \quad \quad \quad \quad \quad \quad \quad \quad \ \ = \eta \\ \\
    & \Rightarrow \big|\rho\big|^{2} = \eta \pm \sqrt{\eta^{2} - 1} \\ \\
    & \quad \quad \quad \ = \eta - \sqrt{\eta^{2} - 1} \\ \\
    & \Rightarrow \big|\rho\big| = \sqrt{\eta - \sqrt{\eta^{2} - 1}} \ , \quad \eta = \big|\Gamma\big|^{2} + \sqrt{\big|\Gamma\big|^{4} - 2\cos\big(2\angle{\Gamma}\big)|\Gamma|^{2} + 1}
  \end{align*}\newline

Similarly, $\angle{\rho}$ can be calculated as a function of $\big|\Gamma\big|$ and $\angle{\Gamma}$:

\begin{equation*}
    \angle{\rho} = \arctan\Bigg( \frac{\tan\big(\angle{\rho}\big)}{\tanh\big(\big|\rho\big|\big)} \Bigg)
  \end{equation*}\newline

Just as a warning, the angle from the equation above has to be wrapped since it'll produce a negative $\angle{\rho}$ for any $\angle{\Gamma}^{\circ} > 90^{\circ}$ (i.e., $\Gamma_{R} < 0$).

\newpage
\section*{Appendix B: Spectral Magnitude of Irreducible Quadratic DTFT}

\begin{align*}
    f(z)&= \big(1 + \rho z^{-1}\big)\big(1 + \rho^{*}z^{-1}\big) \\ \\
        &= 1 + 2\big|\rho\big|\cos\big(\angle{\rho}\big)z^{-1} + \big|\rho\big|^{2}z^{-2} \\ \\
        &=\vec{c}^{T}\begin{bmatrix} z^{-n} \end{bmatrix}^{N} \quad, \quad \vec{c} \equiv \begin{bmatrix}
                                                                                                                       \big|\rho\big|^{2} \\ \\
                                                                                                2\big|\rho\big|\cos\big(\angle{\rho}\big) \\ \\
                                                                                                                                        1
                                                                                            \end{bmatrix} \\
  \end{align*}

\begin{equation*}
    \Rightarrow R_cc[n] = \begin{cases}
                                \quad \quad \quad \quad \quad \quad \quad \quad \ \ \big|\rho\big|^{2} \ , \quad n = 2 \\ \\
                                2\big|\rho\big|\bigg(\big|\rho\big|^{2} + 1\bigg)\cos\big(\angle{\rho}\big) \ , \quad n = 1 \\ \\
                                1 + 4\big|\rho\big|\cos^{2}\big(\angle{\rho}\big) + \big|\rho\big|^{4} \ , \quad n = 0
                            \end{cases}
  \end{equation*}

\begin{align*}
    \Rightarrow \Big|f\big(\omega\big)\Big|^{2}&= R_{cc}[0] + 2\sum_{n = 1}^{N - 1}R_{cc}[n]T_{n}\Big(\cos\big(\omega\big)\Big) \\ \\
                                               &= R_{cc}[0] + 2\begin{bmatrix}
                                                                   2R_{cc}[2] \\ \\
                                                                    R_{cc}[1] \\ \\
                                                                   -R_{cc}[2]
                                                                 \end{bmatrix}^{T}
                                                               \begin{bmatrix}
                                                                   \cos^{2}\big(\omega\big) \\ \\
                                                                       \cos\big(\omega\big) \\ \\
                                                                                          1
                                                                 \end{bmatrix} \\ \\
                                               &= 4R_{cc}[2]\cos^{2}\big(\omega\big) + 2R_{cc}[1]\cos\big(\omega\big) + R_{cc}[0] - 2R_{cc}[2] \\ \\
                                               &= 4R_{cc}[2]\Bigg(\cos\big(\omega\big) + \frac{R_{cc}[1]}{4R_{cc}[2]}\Bigg)^{2} + R_{cc}[0] - 2R_{cc}[2] - \frac{R_{cc}^{2}[1]}{4R_{cc}[2]} \\ \\
                                               &= 4\big|\rho\big|^{2}\Bigg(\cos\big(\omega\big) + \cos\big(\angle{\rho}\big)\cosh\Big(\ln\big[\big|\rho\big|\big]\Big)\Bigg)^{2} + R_{cc}[0] - 2R_{cc}[2] - \frac{R_{cc}^{2}[1]}{4R_{cc}[2]} \\ \\
                                               &= 4\big|\rho\big|^{2}\Bigg(\cos\big(\omega\big) + \cos\big(\angle{\rho}\big)\cosh\Big(\ln\big[\big|\rho\big|\big]\Big)\Bigg)^{2} + 4\big|\rho\big|^{2}\sin^{2}\big(\angle{\rho}\big)\sinh^{2}\Big(\ln\big[\big|\rho\big|\big]\Big) \\ \\
                                               &= K\bigg[\Big(\cos\big(\omega\big) + \Gamma_{R}\Big)^{2} + \Gamma_{I}^{2}\bigg] \\ \\
                                               &= K\Big[\cos\big(\omega\big) + \Gamma\Big]\Big[\cos\big(\omega\big) + \Gamma^{*}\Big]
  \end{align*}

\newpage
\section*{Appendix C: Spectral Magnitude of General DTFT}

\begin{align*}
    &\quad \ \ f(z) = \sum_{n = 0}^{N - 1}c_{n}z^{-n} \\ \\
    &\Rightarrow f\big(\omega\big) = \sum_{n = 0}^{N - 1}c_{n}e^{-jn\omega} \\ \\
    &\quad \quad \quad \quad = \vec{c}^{T}\begin{bmatrix} q[n] \end{bmatrix}^{N} \quad, \quad \vec{c} \equiv = \begin{bmatrix}
                                                                                                                    c_{N - 1} \\
                                                                                                                    c_{N - 2} \\
                                                                                                                    c_{N - 3} \\
                                                                                                                       \vdots \\
                                                                                                                    c_{    2} \\
                                                                                                                    c_{    1} \\
                                                                                                                    c_{    0}
                                                                                                                 \end{bmatrix} \quad, \quad q[n] = e^{-jn\omega} \\ \\
    &\Rightarrow \Big|f\big(\omega\big)\Big|^{2} = f\big(\omega\big)f^{*}\big(\omega\big) \\ \\
    &\quad \quad \quad \quad \quad = \bigg(\vec{c}^{T}\begin{bmatrix} q[n] \end{bmatrix}^{N}\bigg)\bigg(\vec{c}^{T}\begin{bmatrix} q[n] \end{bmatrix}^{N}\bigg)^{H} \\ \\
    &\quad \quad \quad \quad \quad = \vec{c}^{T}\begin{bmatrix} q[n] \end{bmatrix}^{N}\prescript{}{N}{\begin{bmatrix} q^{*}[n] \end{bmatrix}}\vec{c} \\ \\
    &\quad \quad \quad \quad \quad = \vec{c}^{T}\Bigg(I_{N} + \sum_{n = 1}^{N - 1}\bigg( e^{-jn\omega}R_{N, n} + e^{ jn\omega}R_{N, n}^{T} \bigg)\Bigg)\vec{c} \\ \\
    &\quad \quad \quad \quad \quad = \vec{c}^{T}\vec{c} + \sum_{n = 1}^{N - 1}\bigg( e^{-jn\omega}\vec{c}^{T}R_{N, n}\vec{c} + e^{ jn\omega}\vec{c}^{T}R_{N, n}^{T}\vec{c} \bigg) \\ \\
    &\quad \quad \quad \quad \quad = \big|\big|\vec{c}\big|\big|^{2} + \sum_{n = 1}^{N - 1}\bigg( e^{-jn\omega}\vec{c}^{T}R_{N, n}\vec{c} + e^{ jn\omega}\vec{c}^{T}R_{N, n}\vec{c} \bigg) \\ \\
    &\quad \quad \quad \quad \quad = \big|\big|\vec{c}\big|\big|^{2} + \sum_{n = 1}^{N - 1}\vec{c}^{T}R_{N, n}\vec{c}\Big( e^{-jn\omega} + e^{ jn\omega} \Big) \\ \\
    &\quad \quad \quad \quad \quad = \big|\big|\vec{c}\big|\big|^{2} + \sum_{n = 1}^{N - 1}R_{cc}[n]\Big( 2\cos\big(n\omega\big) \Big) \\ \\
    &\quad \quad \quad \quad \quad = R_{cc}[0] + 2\sum_{n = 1}^{N - 1}R_{cc}[n]T_{n}\Big(\cos\big(\omega\big)\Big)
  \end{align*}

\newpage
\section*{Appendix D: Complex Multi-Dimensional Matrix Recurrence Relations}

Consider the complex causal sequence $q[n]$ and its implicit scalar recurrence relation:

\begin{align*}
    q[n]&= e^{-jn\omega}\mu[n] \\ \\
        &= \delta[n] + e^{-j\omega}q[n - 1]
  \end{align*}\newline

The sequence generated by $e^{-jn\omega}$ can then be decomposed into three individual components:

\begin{align*}
    e^{-jn\omega}&= e^{-jn\omega}\bigg(\mu\big[-(n + 1)\big] + \mu[n]\bigg) \\ \\
                 &= e^{-jn\omega}\mu\big[-(n + 1)\big] + e^{-jn\omega}\mu[n] \\ \\
                 &= e^{j\omega}q^{*}\big[-(n + 1)\big] + q[n] \\ \\
                 &= e^{j\omega}q^{*}\big[-(n + 1)\big] + \delta[n] + e^{-j\omega}q[n - 1] \\ \\
                 &= \delta[n] + e^{-j\omega}q[n - 1] + e^{j\omega}q^{*}\big[-(n + 1)\big]
  \end{align*}\newline

The following outer product can then be expanded via this recurrence relation:

\begin{align*}
    & \quad \ \vec{q}[n] = \begin{bmatrix} q[n] \end{bmatrix}^{N} \quad \quad \quad () \\ \\
    & \Rightarrow \vec{q}[n]\vec{q}^{H}[n] = \begin{bmatrix} q[n] \end{bmatrix}^{N}\prescript{}{N}{\begin{bmatrix} q^{*}[n] \end{bmatrix}} \\ \\
    & \quad \quad \quad \quad \quad \ \ = \prescript{}{N}{\begin{bmatrix} q[n]q^{*}[m] \end{bmatrix}}^{N} \\ \\
    & \quad \quad \quad \quad \quad \ \ = \prescript{}{N}{\begin{bmatrix} q[n - m] \end{bmatrix}}^{N} \\ \\
    & \quad \quad \quad \quad \quad \ \ = \prescript{}{N}{\begin{bmatrix} \delta[n - m] + e^{-j\omega}q[n - m - 1] + e^{j\omega}q^{*}\big[-(n - m + 1)\big]\end{bmatrix}}^{N} \\ \\
    & \quad \quad \quad \quad \quad \ \ = \prescript{}{N}{\begin{bmatrix} \delta[n - m] \end{bmatrix}}^{N} + \big(e^{-j\omega}\big)\prescript{}{N}{\begin{bmatrix} q[n - m - 1] \end{bmatrix}}^{N} + \big(e^{j\omega}\big)\prescript{}{N}{\begin{bmatrix} q^{*}\big[-(n - m + 1)\big]\end{bmatrix}}^{N} \\ \\
    & \quad \quad \quad \quad \quad \ \ = I_{N} + e^{-j\omega}\begin{bmatrix}
                                                                    \vec{0} & \prescript{}{N - 1}{\begin{bmatrix} q[n - m] \end{bmatrix}}^{N - 1} \\ \\
                                                                         0  & \vec{0}^{T}
                                                                \end{bmatrix}
                                                + e^{ j\omega}\begin{bmatrix}
                                                                                                                                          \vec{0}^{T} &      0 \\ \\
                                                                    \prescript{}{N - 1}{\begin{bmatrix} q^{*}\big[-(n - m)\big]\end{bmatrix}}^{N - 1} & \vec{0}
                                                                \end{bmatrix} \\ \\
    & \quad \quad \quad \quad \quad \ \ = I_{N} + e^{-j\omega}\begin{bmatrix}
                                                                    \vec{0} & \prescript{}{N - 1}{\begin{bmatrix} q[n - m] \end{bmatrix}}^{N - 1} \\ \\
                                                                         0  & \vec{0}^{T}
                                                                \end{bmatrix}
                                                + e^{ j\omega}\begin{bmatrix}
                                                                    \vec{0} & \prescript{}{N - 1}{\begin{bmatrix} q[n - m]\end{bmatrix}}^{N - 1} \\ \\
                                                                         0  & \vec{0}^{T}
                                                                \end{bmatrix}^{H}
  \end{align*}\newpage

\begin{align*}
    &= I_{N} + e^{-j\omega}\begin{bmatrix}
                                \vec{0} & I_{N - 1} + e^{-j\omega}\prescript{}{N - 1}{\begin{bmatrix} q[n - m - 1] \end{bmatrix}}^{N - 1} \\ \\
                                     0  & \vec{0}^{T}
                             \end{bmatrix} \\ \\
    &\quad \quad \quad + e^{ j\omega}\begin{bmatrix}
                                            \vec{0} & I_{N - 1} + e^{-j\omega}\prescript{}{N - 1}{\begin{bmatrix} q[n - m - 1] \end{bmatrix}}^{N - 1} \\ \\
                                                 0  & \vec{0}^{T}
                                       \end{bmatrix}^{H} \\ \\
    &= I_{N} + e^{-j\omega}\begin{bmatrix}
                                \vec{0} & I_{N - 1} \\ \\
                                     0  & \vec{0}^{T}
                             \end{bmatrix} +
              e^{-2j\omega}\begin{bmatrix}
                                \vec{0} & \prescript{}{N - 1}{\begin{bmatrix} q[n - m - 1] \end{bmatrix}}^{N - 1} \\ \\
                                     0  & \vec{0}^{T}
                             \end{bmatrix} \\ \\
    &\quad \quad \quad + e^{j\omega}\begin{bmatrix}
                                           \vec{0} & I_{N - 1} \\ \\
                                                0  & \vec{0}^{T}
                                      \end{bmatrix}^{T} 
                      + e^{2j\omega}\begin{bmatrix}
                                           \vec{0} & \prescript{}{N - 1}{\begin{bmatrix} q[n - m - 1] \end{bmatrix}}^{N - 1} \\ \\
                                                0  & \vec{0}^{T}
                                      \end{bmatrix}^{H} \\ \\
    &= I_{N} + e^{-j\omega}\begin{bmatrix}
                                0_{N - 1, 1} & I_{N - 1} \\ \\
                                0_{    1, 1} & 0_{1, N - 1}
                             \end{bmatrix} +
              e^{-2j\omega}\begin{bmatrix}
                                0_{N - 2, 2} & \prescript{}{N - 2}{\begin{bmatrix} q[n - m] \end{bmatrix}}^{N - 2} \\ \\
                                0_{    2, 2} & 0_{2, N - 2}
                             \end{bmatrix} \\ \\
    &\quad \quad \quad + e^{j\omega}\begin{bmatrix}
                                           0_{N - 1, 1} & I_{N - 1} \\ \\
                                           0_{    1, 1} & 0_{1, N - 1}
                                      \end{bmatrix}^{T} 
                      + e^{2j\omega}\begin{bmatrix}
                                           0_{N - 2, 2} & \prescript{}{N - 2}{\begin{bmatrix} q[n - m] \end{bmatrix}}^{N - 2} \\ \\
                                           0_{    2, 2} & 0_{2, N - 2}
                                      \end{bmatrix}^{H} \\ \\
    &= I_{N} + e^{-j\omega}\begin{bmatrix}
                                0_{N - 1, 1} & I_{N - 1} \\ \\
                                0_{    1, 1} & 0_{1, N - 1}
                             \end{bmatrix} +
              e^{-2j\omega}\begin{bmatrix}
                                0_{N - 2, 2} & I_{N - 2} \\ \\
                                0_{    2, 2} & 0_{2, N - 2}
                             \end{bmatrix} +
              e^{-3j\omega}\begin{bmatrix}
                                0_{N - 3, 2} & \prescript{}{N - 3}{\begin{bmatrix} q[n - m] \end{bmatrix}}^{N - 3} \\ \\
                                0_{    3, 2} & 0_{3, N - 3}
                             \end{bmatrix}  \\ \\
    &\quad \quad \quad + e^{j\omega}\begin{bmatrix}
                                           0_{N - 1, 1} & I_{N - 1} \\ \\
                                           0_{    1, 1} & 0_{1, N - 1}
                                      \end{bmatrix}^{T} + 
                        e^{2j\omega}\begin{bmatrix}
                                           0_{N - 2, 2} & I_{N - 2} \\ \\
                                           0_{    2, 2} & 0_{2, N - 2}
                                      \end{bmatrix}^{T} + 
                        e^{3j\omega}\begin{bmatrix}
                                           0_{N - 3, 3} & \prescript{}{N - 3}{\begin{bmatrix} q[n - m] \end{bmatrix}}^{N - 3} \\ \\
                                           0_{    3, 3} & 0_{3, N - 3}
                                      \end{bmatrix}^{H} \\ \\
    &= I_{N} + \sum_{n = 1}^{N - 1}\begin{pmatrix}
                                        e^{-jn\omega}\begin{bmatrix}
                                                            0_{N - n, n} & I_{n} \\ \\
                                                            0_{    n, n} & 0_{n, N - n}
                                                       \end{bmatrix} +
                                        e^{ jn\omega}\begin{bmatrix}
                                                            0_{N - n, n} & I_{n} \\ \\
                                                            0_{    n, n} & 0_{n, N - n}
                                                       \end{bmatrix}^{T}
                                     \end{pmatrix} \\ \\
    &= I_{N} + \sum_{n = 1}^{N - 1}\bigg( e^{-jn\omega}R_{N, n} + e^{ jn\omega}R_{N, n}^{T} \bigg)
  \end{align*}

\newpage
\section*{Appendix E: Matrix Representation of Autocorrelation Sequence}

For a given causal sequence $x[n]$, the Autocorrelation is defined as $R_{xx}[n]$:

\begin{align*}
    R_{xx}[l]&= \sum_{n = 0}^{\infty}s[n]s[n + l] \quad, \quad s[n] \equiv x[n]\Big(\mu[n] - \mu[n - N]\Big) \\ \\
             &= \sum_{n = 0}^{N - (l + 1)}x[n]x[n + l] \\ \\
             &= \prescript{}{N}{\begin{bmatrix} x[n] \end{bmatrix}}
                \begin{bmatrix}
                    \vec{0} \\ \\
                    \begin{bmatrix} x[n] \end{bmatrix}^{n = N - 1}_{n = l}
                  \end{bmatrix} \\ \\
             &= \Big( I_{N}\begin{bmatrix} x[n] \end{bmatrix}^{N} \Big)^{T}
                \begin{pmatrix}
                    \begin{bmatrix}
                        0_{l, N - l} & 0_{    l, l} \\
                        I_{   N - l} & 0_{N - l, l}
                      \end{bmatrix}\begin{bmatrix} x[n] \end{bmatrix}^{N}
                  \end{pmatrix} \\ \\
             &=\prescript{}{N}{\begin{bmatrix} x[n] \end{bmatrix}}I_{N}\begin{bmatrix}
                                                                            0_{l, N - l} & 0_{    l, l} \\
                                                                            I_{   N - l} & 0_{N - l, l}
                                                                         \end{bmatrix}\begin{bmatrix} x[n] \end{bmatrix}^{N} \\ \\
             &=\prescript{}{N}{\begin{bmatrix} x[n] \end{bmatrix}}\begin{bmatrix}
                                                                        0_{l, N - l} & 0_{    l, l} \\
                                                                        I_{   N - l} & 0_{N - l, l}
                                                                    \end{bmatrix}\begin{bmatrix} x[n] \end{bmatrix}^{N} \\ \\
             &=\prescript{}{N}{\begin{bmatrix} x[n] \end{bmatrix}}\begin{bmatrix}
                                                                        0_{l, N - l} & I_{N - l} \\
                                                                        0_{l,     l} & 0_{N - l, l}
                                                                    \end{bmatrix}\begin{bmatrix} x[n] \end{bmatrix}^{N} \\ \\
             &=\prescript{}{N}{\begin{bmatrix} x[n] \end{bmatrix}}R_{N, l}\begin{bmatrix} x[n] \end{bmatrix}^{N} \quad, \quad R_{N, l} \equiv \begin{bmatrix}
                                                                                                                                                    0_{l, N - l} & I_{N - l} \\
                                                                                                                                                    0_{l,     l} & 0_{N - l, l}
                                                                                                                                                \end{bmatrix}
  \end{align*}

\newpage
\section*{Appendix F: Custom Notation for Vector/Matrix-generating Sequences}

\hspace{5mm} Some of the recurrence relations in Appendix A rely on the mathematics of vector/matrix-generating sequences. However, there seems to be no standard notation for any kind of vector/matrix generated from such sequences. This Appendix contains some of my own custom notation which I created for expressing these vectors/matrixs in a concise manner.\newline 

Take a generic sequence $f[n]$. The column vector $\vec{v}$ (populated by applying $f[n]$ element-wise to each row in the vector as a function of the row) can be denoted below as such:

\begin{align*}
    \vec{v} = \begin{bmatrix} f[n] \end{bmatrix}^{N}
            = \begin{bmatrix} f[n] \end{bmatrix}^{n=N - 1}_{n=0}
            = \begin{bmatrix}
                    f[N - 1] \\ \\
                    f[N - 2] \\ \\
                    f[N - 3] \\ \\
                      \vdots \\ \\
                    f[    2] \\ \\
                    f[    1] \\ \\
                    f[    0] 
                \end{bmatrix}
  \end{align*}

Accordingly, the row vector $\vec{v}^{T}$ can be denoted below as such:

\begin{align*}
    \vec{v}^{T}&=   \prescript{}{N}{\begin{bmatrix} f[n] \end{bmatrix}} \ 
                = \ \prescript{}{n=N - 1}{\begin{bmatrix} f[n] \end{bmatrix}}_{n=0} \\ \\
               &= \begin{bmatrix} f[N - 1] & f[N - 2] & f[N - 3] & \ldots & f[2] & f[1] & f[0] \end{bmatrix} \\
  \end{align*} 

The matrix $M$ (populated by applying $f[n, m]$ elementwise to each row $n$ and column $m$ in the matrix as a function of $n$ and $m$) can be denoted below as such:

\begin{align*}
    M&=   \prescript{}{N}{\begin{bmatrix} f[n, m] \end{bmatrix}}^{N} \ 
      = \ \prescript{}{m = N - 1}{\begin{bmatrix} f[n, m] \end{bmatrix}}^{n = N - 1}_{n = 0,m =0} \\ \\
     &= \begin{bmatrix}
            f[N - 1, N - 1] & f[N - 1, N - 2] & f[N - 1, N - 3] & \ldots & f[N - 1, 2] & f[N - 1, 1] & f[N - 1, 0] \\ \\
            f[N - 2, N - 1] & f[N - 2, N - 2] & f[N - 2, N - 3] & \ldots & f[N - 2, 2] & f[N - 2, 1] & f[N - 2, 0] \\ \\
            f[N - 3, N - 1] & f[N - 3, N - 2] & f[N - 3, N - 3] & \ldots & f[N - 3, 2] & f[N - 3, 1] & f[N - 3, 0] \\ \\
                     \vdots &          \vdots &          \vdots & \ddots &      \vdots &      \vdots &      \vdots \\ \\
            f[    2, N - 1] & f[    2, N - 2] & f[    2, N - 3] & \ldots & f[    2, 2] & f[    2, 1] & f[    2, 0] \\ \\
            f[    1, N - 1] & f[    1, N - 2] & f[    1, N - 3] & \ldots & f[    1, 2] & f[    1, 1] & f[    1, 0] \\ \\
            f[    0, N - 1] & f[    0, N - 2] & f[    0, N - 3] & \ldots & f[    0, 2] & f[    0, 1] & f[    0, 0]
          \end{bmatrix} \\ \\
  \end{align*} 

\newpage

Sometimes, some matrices will have portions systematically populated by zeroes. In such cases, we can have the original function ($f[n, m]$) modified by another function as a sort of indicator function. For example, consider an upper triangular matrix $U$ where $f[n, m]$ is multiplied by a heaviside step function. $U$ can be denoted as such below:

\begin{align*}
    U&= \prescript{}{N}{\begin{bmatrix} g[n, m] \end{bmatrix}}^{N} \quad , \quad g(n, m) = f[n, m]\mu[n - m] \\ \\
     &= \begin{bmatrix}
            f[N - 1, N - 1] & f[N - 1, N - 2] & f[N - 1, N - 3] & \ldots & f[N - 1, 2] & f[N - 1, 1] & f[N - 1, 0] \\ \\
                          0 & f[N - 2, N - 2] & f[N - 2, N - 3] & \ldots & f[N - 2, 2] & f[N - 2, 1] & f[N - 2, 0] \\ \\
                          0 &               0 & f[N - 3, N - 3] & \ldots & f[N - 3, 2] & f[N - 3, 1] & f[N - 3, 0] \\ \\
                     \vdots &          \vdots &          \vdots & \ddots &      \vdots &      \vdots &      \vdots \\ \\
                          0 &               0 &               0 & \ldots & f[    2, 2] & f[    2, 1] & f[    2, 0] \\ \\
                          0 &               0 &               0 & \ldots &           0 & f[    1, 1] & f[    1, 0] \\ \\
                          0 &               0 &               0 & \ldots &           0 &           0 & f[    0, 0]
          \end{bmatrix} \\ \\
  \end{align*}

The inner product of $\vec{v}$ with itself can be denoted as such below:

\begin{align*}
    \vec{v}^{T}\vec{v} = \prescript{}{N}{\begin{bmatrix} f[n] \end{bmatrix}}\begin{bmatrix} f[n] \end{bmatrix}^{N}
                       = \prescript{}{n = N - 1}{\begin{bmatrix} f[n] \end{bmatrix}}_{n = 0}\begin{bmatrix} f[n] \end{bmatrix}^{n = N - 1}_{n = 0} = \sum^{n = N - 1}_{n = 0}f^{2}[n] \\ \\
  \end{align*}

The outer product of $\vec{v}$ with itself can be denoted as such below:

\begin{align*}
    \vec{v}\vec{v}^{T}&= \begin{bmatrix} f[n] \end{bmatrix}^{N}\prescript{}{N}{\begin{bmatrix} f[n] \end{bmatrix}}
                       = \begin{bmatrix} f[n] \end{bmatrix}^{n = N - 1}_{n = 0}\prescript{}{n = N - 1}{\begin{bmatrix} f[n] \end{bmatrix}}_{n = 0}
                       = \prescript{}{m = N - 1}{\begin{bmatrix} f[n]f[m] \end{bmatrix}}^{n = N - 1}_{n = 0, m = 0} \\ \\ 
                      &= \begin{bmatrix}
                            f[N - 1]f[N - 1] & f[N - 1]f[N - 2] & f[N - 1]f[N - 3] & \ldots & f[N - 1]f[2] & f[N - 1]f[1] & f[N - 1]f[0] \\ \\
                            f[N - 2]f[N - 1] & f[N - 2]f[N - 2] & f[N - 2]f[N - 3] & \ldots & f[N - 2]f[2] & f[N - 2]f[1] & f[N - 2]f[0] \\ \\
                            f[N - 3]f[N - 1] & f[N - 3]f[N - 2] & f[N - 3]f[N - 3] & \ldots & f[N - 3]f[2] & f[N - 3]f[1] & f[N - 3]f[0] \\ \\
                                      \vdots &           \vdots &           \vdots & \ddots &       \vdots &       \vdots &       \vdots \\ \\
                            f[    2]f[N - 1] & f[    2]f[N - 2] & f[    2]f[N - 3] & \ldots & f[    2]f[2] & f[    2]f[1] & f[    2]f[0] \\ \\
                            f[    1]f[N - 1] & f[    1]f[N - 2] & f[    1]f[N - 3] & \ldots & f[    1]f[2] & f[    1]f[1] & f[    1]f[0] \\ \\
                            f[    0]f[N - 1] & f[    0]f[N - 2] & f[    0]f[N - 3] & \ldots & f[    0]f[2] & f[    0]f[1] & f[    0]f[0]
                           \end{bmatrix} \\ \\ 
  \end{align*}

\end{document}
